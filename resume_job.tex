%%%%%%%%%%%%%%%%%%%%%%%%%%%%%%%%%%%%%%%%%%%%%%%%%%%%%%%%%%%%%%%%%%%%%%%%%%%%%%%
% A clean template for an academic CV. This is a short summary version.
%
% Uses tabularx to create two column entries (date and job/edu/citation).
% Defines commands to make adding entries simpler.
%
%%%%%%%%%%%%%%%%%%%%%%%%%%%%%%%%%%%%%%%%%%%%%%%%%%%%%%%%%%%%%%%%%%%%%%%%%%%%%%%

\documentclass[9pt,a4paper]{article}

% Useful aliases
\newcommand{\TJU}{Tongji University}
\newcommand{\TP}{Télécom Paris}
\newcommand{\HDU}{Hangzhou Dianzi University}


% Identifying information
\newcommand{\Title}{Résumé}
\newcommand{\FirstName}{Shuaifan}
\newcommand{\LastName}{Xia}
\newcommand{\Initials}{S}
\newcommand{\MyName}{\FirstName\ \LastName}
\newcommand{\Me}{\underline{\Initials. \LastName}}  % For citations
\newcommand{\Email}{shuaifan.xia@telecom-paris.fr}
\newcommand{\PersonalWebsite}{collin911.github.io/mypage}
\newcommand{\LabWebsite}{www.compgeolab.org}
\newcommand{\ORCID}{0000-0002-9769-4373}
\newcommand{\GitHubProfile}{collin911}

% Names for citing coauthors
\newcommand{\Lqw}{Q. Liu}
\newcommand{\Lmq}{M. Liu}
\newcommand{\Fw}{W. Fang}
\newcommand{\Jqw}{Q. Jiang}
\newcommand{\Lxz}{X. Li}
\newcommand{\Xml}{M. Xiong}
\newcommand{\Zsl}{S. Zhou}
\newcommand{\Xmy}{M. Xu}
\newcommand{\Dh}{H. Deng}



% Load packages
%%%%%%%%%%%%%%%%%%%%%%%%%%%%%%%%%%%%%%%%%%%%%%%%%%%%%%%%%%%%%%%%%%%%%%%%%%%%%%%

% Full Unicode support for non-ASCII characters
\usepackage[utf8]{inputenc}
\usepackage[english]{babel}
\usepackage[TU]{fontenc}

% Set main fonts
\usepackage[sfdefault]{atkinson}
\usepackage[ttdefault]{sourcecodepro}

% Icon fonts
\usepackage{fontawesome5}
\usepackage{academicons}

% Disable hyphenation
\usepackage[none]{hyphenat}

% Control the font size
\usepackage{anyfontsize}

% For fancy and multipage tables
\usepackage{tabularx}
\usepackage{ltablex}

% For new environments
\usepackage{environ}

% Manage dates and times
\usepackage{datetime}

% Set the page margins
\usepackage{geometry}

% To get the total page numbers (\pageref{LastPage})
\usepackage{lastpage}

% Control spacing in enumerates
\usepackage{enumitem}

% Use custom colors
\usepackage[usenames,dvipsnames]{xcolor}

% Configure section titles
\usepackage{titlesec}

% Fancy header configuration
\usepackage{fancyhdr}

% Control PDF metadata and links
\usepackage[colorlinks=true]{hyperref}


% Template configuration
%%%%%%%%%%%%%%%%%%%%%%%%%%%%%%%%%%%%%%%%%%%%%%%%%%%%%%%%%%%%%%%%%%%%%%%%%%%%%%%

\geometry{%
  margin=12.5mm,
  headsep=0mm,
  headheight=0mm,
  footskip=5mm,
  includehead=true,
  includefoot=true
}

% Custom colors
\definecolor{mediumgray}{gray}{0.5}
\definecolor{lightgray}{gray}{0.9}
\definecolor{mediumblue}{HTML}{2060c2}
\definecolor{lightblue}{HTML}{a0c3ff}

% No indentation
\setlength\parindent{0cm}

% Increase the line spacing
\renewcommand{\baselinestretch}{1.1}
% and the spacing between rows in tables
\renewcommand{\arraystretch}{1.25}

% Remove space between items in itemize and enumerate
\setlist{nosep}

% Set the spacing and format of sections
\titleformat{\section}
  {\normalfont\Large\mdseries} % format
  {} % label
  {0pt} % separation (left separation for hang)
  {} % text before title
  [\titlerule] % text after title
\titlespacing*{\section}
  {0pt} % left pad
  {0.1cm} % before
  {0cm} % after

% Disable number of sections. Use this instead of "section*" so that the sections still
% appear as PDF bookmarks. Otherwise, would have to add the table of contents entries
% manually.
\makeatletter
\renewcommand{\@seccntformat}[1]{}
\makeatother

% Define a new environment to place all CV entries in a 2-column table.
% Left column are the dates, right column the entries.
\newcommand{\TablePad}{\vspace{-0.2cm}}
\NewEnviron{EntriesTableDuration}{
\TablePad
\begin{tabularx}{\textwidth}{@{}p{0.10\textwidth}@{\hspace{0.02\textwidth}}p{0.88\textwidth}@{}}
  \BODY
\end{tabularx}
\TablePad
}
\NewEnviron{EntriesTableYear}{
\TablePad
\begin{tabularx}{\textwidth}{@{}p{0.05\textwidth}@{\hspace{0.01\textwidth}}p{0.94\textwidth}@{}}
  \BODY
\end{tabularx}
\TablePad
}

% Macros to set the year and duration on the left column
\newcommand{\Duration}[2]{\fontsize{10pt}{0}\selectfont \texttt{#1-#2}}
\newcommand{\Year}[1]{\fontsize{10pt}{0}\selectfont \texttt{#1}}
\newcommand{\Ongoing}{on}
\newcommand{\Future}{future}

% Macros to add links and mark publications
\newcommand{\DOI}[1]{doi:\href{https://doi.org/#1}{#1}}
\newcommand{\Website}[1]{\href{https://#1}{#1}}
\newcommand{\Preprint}[1]{Preprint: \href{https://doi.org/#1}{#1}}
\newcommand{\GitHub}[1]{\faGithub{} \href{https://github.com/#1}{#1}}
\newcommand{\Data}[1]{\faChartBar{} doi:\href{https://doi.org/#1}{#1}}

% Define command to insert month name and year as date
\newdateformat{monthyear}{\monthname[\THEMONTH], \THEYEAR}

% Configure a fancy footer
\newcommand{\Separator}{\hspace{3pt}|\hspace{3pt}}
\newcommand{\FooterFont}{\footnotesize\color{mediumgray}}
\pagestyle{fancy}
\fancyhf{}
\lfoot{%
  \FooterFont{}
  \MyName{}
  \Separator{}
  \Title{}
}
\rfoot{%
  \FooterFont{}
  Last updated: \monthyear\today{}
  \Separator{}
  \thepage\space of\space \pageref*{LastPage}
}
\renewcommand{\headrulewidth}{0pt}
\renewcommand{\footrulewidth}{1pt}
\preto{\footrule}{\color{lightgray}}

% Metadata for the PDF output and control of hyperlinks
\hypersetup{
  colorlinks,
  allcolors=mediumblue,
  breaklinks=true,
  pdftitle={\Title{} - \MyName},
  pdfauthor={\MyName},
}

%%%%%%%%%%%%%%%%%%%%%%%%%%%%%%%%%%%%%%%%%%%%%%%%%%%%%%%%%%%%%%%%%%%%%%%%%%%%%%%
\begin{document}

\begin{minipage}[t]{0.5\textwidth}
  {\fontsize{20pt}{0}\selectfont\MyName}
\end{minipage}
\begin{minipage}[t]{0.5\textwidth}
  \begin{flushright}
    \Title{}
  \end{flushright}
\end{minipage}
\\[-0.1cm]
\textcolor{lightgray}{\rule{\textwidth}{3pt}}
\begin{minipage}[t]{0.75\textwidth}
  Seek for Stage/CDI/CIFRES in 5G/6G Wireless Communications
  \\
  Website: \Website{\PersonalWebsite}
  \\
  Email: \href{mailto:\Email}{\Email}
  \\
  
\end{minipage}
\begin{minipage}[t]{0.07\textwidth}
  %\begin{flushright}
    \textbf{Chinese} 
    \\
    \textbf{English} 
    \\
    \textbf{French} 
    \\
    %\textbf{German}
  %\end{flushright}
\end{minipage}
\begin{minipage}[t]{0.18\textwidth}
  \begin{flushright}
    Native Speaker
    \\
    TOEFL 107
    \\
    B1
    \\
    %Basic
  \end{flushright}
\end{minipage}


%%%%%%%%%%%%%%%%%%%%%%%%%%%%%%%%%%%%%%%%%%%%%%%%%%%%%%%%%%%%%%%%%%%%%%%%%%%%%%%
\section{Education}

\begin{EntriesTableDuration}
  \Duration{2024}{\Ongoing}  &
  \textbf{Ingénieur in Computer Science}, \TP, Île-de-France, France.
  \\ & GPA: 3.93/4.0
  %\DOI{10.6084/m9.figshare.16883689}
  \\
  \Duration{2022}{2025}  &
  \textbf{M.Eng in Computer Science}, \TJU, Shanghai, China.
  \\ & GPA: 4.83/5.0 Ranking: 1/48 \emph{Summa Cum Laude}
  % \\ & 
  Research Area: Internet of Things
  %\DOI{10.6084/m9.figshare.16882300}
  \\
  \Duration{2018}{2022}  &
  \textbf{B.Eng in Information Security}, \HDU, Zhejiang, China.
  \\ & GPA: 4.72/5.0 Ranking: 1/116 \emph{Summa Cum Laude}
  % \\ & Core Courses: Computer Networks, Principles of Communication, Signals and Systems 
  %\DOI{10.6084/m9.figshare.963547}
\end{EntriesTableDuration}

\section{Projects \& Grants}

\begin{EntriesTableDuration}
  \Duration{2022}{\Ongoing} &
  \textbf{Resonant Beam System} %| \Website{www.fatiando.org}
  \hfill Project member, researcher, and simulation program developer.
  \newline
  Employing intra-cavity laser resonance to achieve wireless power transfer, communication, and positioning. 
  \newline Providing interpretable model and simulation program based on Fourier optics for system design and performance evaluation.
  \newline Analyzing the channel characteristics and transmission performance.
  \newline Exploring the possibility of system extension with mmWave and RIS/IRS.
  \newline \textbf{Major Skills}: Python, MATLAB, Wave and Optical Engineering, Communication Engineering.
  \newline 
  \textbf{Related publications} (Reference Google Scholar: \href{https://scholar.google.com/citations?user=x4Y8uCAAAAAJ}{\MyName}): 
  \newline 
  6 (Co-)Authored Journal Papers (IEEE Trans. Wireless Comm., IEEE IoT. J.).
  \newline
  2 Patents (China) under examination as to substance.
  \\

  \Duration{2019}{2022} &
  \textbf{Insight Security} \hfill Role: Project leader, system architect, and developer.
  \newline
  Developed a comprehensive security framework for industrial and IoT networks integrating firewalls, intrusion detection systems (IDS), and vulnerability assessment modules.
	\newline
  Implemented a cloud–edge–terminal collaborative architecture enabling coordinated threat detection and adaptive response across distributed environments.
	\newline
  Designed a real-time visualization and control platform to monitor network status, intrusion alerts, and system performance.
	\newline \textbf{Major Skills}: Distributed software dev. in Java and C/C++, ELK, Network Security.
  \newline 
  \textbf{Related awards and grants}:
  \newline 
  2021 China College Computer Competition (C4) 
  \hfill National Second Prize 
  \newline 
  2020 National Innovation Project Grants for College Students
  \hfill National Grant
  \newline 
  2019 National Innovation Project Grants for College Students
  \hfill National Grant
  \newline
  2 Patents (China) Granted.
  \\

\Duration{2021}{2021} & 
    \textbf{Interdisciplinary Contest in Modeling (ICM)}.
    \hfill Role: Team leader, modeller, and paper writer.
    \newline
    Topic: Exploration of the Impact Factors for Music Trending - A Graph-based Approach.
    \newline
    Honorable Mention, USA.
  \\
\end{EntriesTableDuration}

\iffalse
%%%%%%%%%%%%%%%%%%%%%%%%%%%%%%%%%%%%%%%%%%%%%%%%%%%%%%%%%%%%%%%%%%%%%%%%%%%%%%%
\section{Community Service}

\begin{EntriesTableDuration}
  \Duration{2024}{\Ongoing} & \textbf{Embaixador}, Rede Brasileira de Reprodutibilidade, \Website{www.reprodutibilidade.org}
  \\
  \Duration{2024}{\Ongoing} & \textbf{Advisory Council Member}, EarthArXiv, \Website{eartharxiv.org}
  \\
  \Duration{2022}{\Ongoing} & \textbf{Board Member}, Software Underground, \Website{softwareunderground.org}
  \\
  \Duration{2022}{2023} & \textbf{Advisory Committee Member}, pyOpenSci, \Website{www.pyopensci.org}
  \\
  \Duration{2019}{2022} & \textbf{Topic Editor}, Journal of Open Source Software, \Website{joss.theoj.org}
\end{EntriesTableDuration}


%%%%%%%%%%%%%%%%%%%%%%%%%%%%%%%%%%%%%%%%%%%%%%%%%%%%%%%%%%%%%%%%%%%%%%%%%%%%%%%
\section{Open Research Software}

\begin{EntriesTableDuration}
  \Duration{2010}{\Ongoing} &
  \textbf{Fatiando a Terra} | \Website{www.fatiando.org}
  \newline
  \textit{Python tools for geophysical data processing, forward modeling, and inversion}
  \newline
  Role: Project founder, core developer, Steering Council Member
  \\
  \Duration{2017}{\Ongoing} &
  \textbf{The Generic Mapping Tools (GMT)} | \Website{www.generic-mapping-tools.org}
  \newline
  \textit{A data processing and mapping toolbox for the Earth, Ocean, and Planetary Science}
  \newline
  Role: Community stewardship advisor, set up the website + forum + GitHub workflow
  \\
  \Duration{2022}{\Ongoing} &
  \textbf{xlandsat} | \Website{www.compgeolab.org/xlandsat}
  \newline
  \textit{Load Landsat remote sensing scenes in Python and xarray}
  \newline
  Role: Creator and sole developer
  \\
  \Duration{2017}{2021} &
  \textbf{PyGMT} | \Website{www.pygmt.org}
  \newline
  \textit{A Python interface for the Generic Mapping Tools}
  \newline
  Role: Project founder, developer, advisor
  \\
  \Duration{2009}{2016} &
  \textbf{Tesseroids} | \Website{tesseroids.leouieda.com}
  \newline
  \textit{Forward modeling of gravitational fields in spherical coordinates}
  \newline
  Role: Creator and sole developer
\end{EntriesTableDuration}

%%%%%%%%%%%%%%%%%%%%%%%%%%%%%%%%%%%%%%%%%%%%%%%%%%%%%%%%%%%%%%%%%%%%%%%%%%%%%%%
\section{Open Educational Resources}

\begin{EntriesTableYear}
  \Year{2022} &
  \textbf{A Quick Introduction to Machine Learning}.
  \GitHub{leouieda/ml-intro}.
  \\
  \Year{2023} &
  \textbf{Remote Sensing with Python}.
  \GitHub{leouieda/remote-sensing}.
  \\
  \Year{2023} &
  \textbf{Lithosphere Dynamics with Python}.
  \GitHub{leouieda/lithosphere}.
  \\
  \Year{2022} &
  \textbf{Terrestrial Gravimetry with Python}.
  \GitHub{leouieda/gravity-processing}.
\end{EntriesTableYear}

%%%%%%%%%%%%%%%%%%%%%%%%%%%%%%%%%%%%%%%%%%%%%%%%%%%%%%%%%%%%%%%%%%%%%%%%%%%%%%%
\section{Grants and Fellowships}

\begin{EntriesTableDuration}
  \Duration{2022}{\Ongoing}  &
  \textbf{Towards individual-grain paleomagnetism: Translating regional-scale geophysics to the nascent field of magnetic microscopy}.
  \newline
  Royal Society.
  \Me{} (PI); \Ricardo{}.
  Award: \href{https://www.compgeolab.org/news/rsoc-mag-microscopy-2022.html}{IES\textbackslash{}R3\textbackslash{}213141}
  \\
  \Duration{2020}{\Ongoing}  &
  \textbf{A Sustainable Plan for the Future of the Generic Mapping Tools}.
  \newline
  NSF-EAR.
  \Paul{} (PI); \Me{}.
  Award: \href{https://www.nsf.gov/awardsearch/showAward?AWD_ID=1948602}{1948602}.
  \\
  \Duration{2020}{2023}  &
  \textbf{SSI Fellowship Programme}.
  \newline
  Software Sustainability Institute.
  \Me{} (PI).
  Award: \Website{software.ac.uk/about/fellows}
  \\
  \Duration{2018}{2024}  &
  \textbf{The EarthScope/GMT Analysis and Visualization Toolbox}.
  \newline
  NSF-EAR.
  \Paul{} (PI); \Me{}; \Bridget{}.
  Award: \href{https://www.nsf.gov/awardsearch/showAward?AWD_ID=1829371}{1829371}.
\end{EntriesTableDuration}

%%%%%%%%%%%%%%%%%%%%%%%%%%%%%%%%%%%%%%%%%%%%%%%%%%%%%%%%%%%%%%%%%%%%%%%%%%%%%%%
\section{Selected Invited Presentations}

\begin{EntriesTableYear}
\Year{2021}  &
  \textbf{Design useful tools that do one thing well and work together: rediscovering the UNIX philosophy while building the Fatiando a Terra project}.
  \newline
  AGU 2021.
  \Me; \LLi; \Santiago; \Agustina.
  \GitHub{fatiando/agu2021}.
  \\
  &
  \textbf{Open-science for gravimetry: tools, challenges, and opportunities}.
  \newline
  GFZ Helmholtz Centre Potsdam.
  \Me; \Santiago; \Agustina.
  \GitHub{leouieda/2021-06-22-gfz}.
  \\
  &
  \textbf{Fatiando a Terra: Open-source tools for geophysics}.
  \newline
  Geophysical Society of Houston.
  \Me; \Santiago; \Agustina.
  \GitHub{fatiando/2021-gsh}.
  \\
\Year{2020}  &
  \textbf{Geophysical research powered by open-source}.
  \newline
  Christian Albrechts Universität zu Kiel.
  \Me.
  \GitHub{leouieda/2020-07-01-kiel}.
\end{EntriesTableYear}
\fi
%%%%%%%%%%%%%%%%%%%%%%%%%%%%%%%%%%%%%%%%%%%%%%%%%%%%%%%%%%%%%%%%%%%%%%%%%%%%%%%

%%%%%%%%%%%%%%%%%%%%%%%%%%%%%%%%%%%%%%%%%%%%%%%%%%%%%%%%%%%%%%%%%%%%%%%%%%%%%%%

%%%%%%%%%%%%%%%%%%%%%%%%%%%%%%%%%%%%%%%%%%%%%%%%%%%%%%%%%%%%%%%%%%%%%%%%%%%%%%%
\section{Honors \& Awards}

\begin{EntriesTableYear}
  \Year{2024} & 
    \textbf{France Excellence EIFFEL Scholarship}.
    \hfill Campus France, France.
  \\
  \Year{2024} & 
    \textbf{National Scholarship for Master Students}.
    \hfill Ministry of Education, China.
  \\
  \Year{2023} & 
    \textbf{National Scholarship for Master Students}.
    \hfill Ministry of Education, China.
  \\
  \Year{2021} & 
    \textbf{Zhejiang Provincial Scholarship}.
    \hfill Department of Education, Zhejiang, China.
  \\
  \Year{2020} & 
    \textbf{National Scholarship for Undergraduate Students}.
    \hfill Ministry of Education, China.
  \\
  \Year{2019} & 
    \textbf{Zhejiang Provincial Scholarship}.
    \hfill Department of Education, Zhejiang, China.
\end{EntriesTableYear}
\newpage
%%%%%%%%%%%%%%%%%%%%%%%%%%%%%%%%%%%%%%%%%%%%%%%%%%%%%%%%%%%%%%%%%%%%%%%%%%%%%%%
\section{Publication Highlights}

\begin{EntriesTableYear}
  \Year{2025}  &
    \textbf{Frequency Division Duplexing Resonant Beam Communication}.
    \newline
    \Me, \Lqw, \Jqw, \Fw, \Lmq.
    \hfill IEEE Transactions on Wireless Communications.
    \newline
    Accepted for publication.
    %|
    %\Data{10.6084/m9.figshare.22672978}
    \newline
    $\bullet$ \emph{Proposed a self-aligned millimeter-wave resonant beam system leveraging retro-directive antenna arrays to achieve adaptive resonance without beam scanning or steering.}
    \newline
    $\bullet$ \emph{Introduced a dual-frequency FDD design to eliminate interference, enabling microsecond-scale resonance establishment and achieving 4.8 bps/Hz spectral efficiency with low bit error rates in indoor scenarios.}
    \\
  \Year{2025}  &
    \textbf{Mobile Self-Protection Resonant Beam SWIPT with Adaptive Phase Control}.
    \newline
    X. Wang, \Me, \Fw, \Lmq, \Xml, \Lqw, Z. Pan.
    \hfill IEEE Internet of Things Journal.
    \DOI{10.1109/JIOT.2025.3595641}
    \newline
    $\bullet$ \emph{Proposed a self-adaptive resonant beam-based SWIPT system with enhanced safety and mobility using a Pockels-effect phase adjuster and optical phase-locked loop for automatic phase correction.}
    \newline
    $\bullet$ \emph{Demonstrated stable performance with 13.77 bps/Hz data rate and 4.63 W power transfer over 6 m while maintaining intrinsic human safety within a 5° field of view.}
    \\
  \Year{2024}  &
    \textbf{Millimeter-wave Resonant Beam SWIPT}.
    \newline
    \Me, \Jqw, \Fw, \Lqw, \Zsl, \Lmq, \Xml.
    \hfill IEEE Internet of Things Journal.
    \DOI{10.1109/JIOT.2024.3452121}
    Open science:
    \GitHub{Collin911/RF-RBSWIPT}
    %|
    %\Data{10.6084/m9.figshare.22672978}
    \newline
    $\bullet$ \emph{Proposed a millimeter-wave resonant beam-based SWIPT system that achieves automatic beam alignment and efficient bidirectional transmission using retro-directive antenna arrays and a dual-frequency design.}
    \newline
    $\bullet$ \emph{Demonstrated through analysis that the system can deliver watt-level wireless power and achieve 4.8 bps/Hz spectral efficiency in indoor environments.}
    \\
  \Year{2023}  &
    \textbf{Auto-Protection for Resonant Beam SWIPT in 
    Portable Applications}.
    \newline
    \Me, \Lqw, \Lmq, \Fw, \Xml, Y. Bai, \Lxz.
    \hfill IEEE Internet of Things Journal.
    \DOI{10.1109/JIOT.2023.3298521}
    Open science:
    \GitHub{Collin911/Resonant-Beam-AutoPro}
    %|
    %\Data{10.6084/m9.figshare.22672978}
    \newline
    $\bullet$ \emph{Developed a portable auto-protection scheme for resonant beam-based SWIPT systems to ensure intrinsic human safety without sacrificing performance.}
    \newline
    $\bullet$ \emph{Proposed a phase compensation and analytical modeling approach enabling automatic beam cutoff upon obstruction, while achieving 13.55 bps/Hz data rate and 5.42 W power transfer over a 6 m range.}
    \\
  \Year{2023}  &
    \textbf{NLOS Transmission Analysis for Mobile SLIPT Using Resonant Beam}.
    \newline
    \Lmq, \Me, \Xml, \Xmy, \Lqw, \Dh.
    \hfill IEEE Transactions on Wireless Communications.
    \DOI{10.1109/TWC.2023.3277593}
    \newline
    $\bullet$ \emph{Developed analytical models and efficient simulation tools for reflector-assisted non-line-of-sight (NLOS) transmission in resonant beam systems, enabling accurate modeling of mobile transmission channels.}
    \newline
    $\bullet$ \emph{Demonstrated that RB-SLIPT can simultaneously achieve 4 W wireless charging power and 12 bit/s/Hz data rate over 2 m NLOS transmission.}
    \\
  \Year{2022}  &
    \textbf{Integrated Communication and Positioning With Resonant Beam}.
    \newline
    \Lmq, \Me, \Xml, \Xmy, \Fw, \Lqw.
    \hfill IEEE Transactions on Wireless Communications.
    \DOI{10.1109/TWC.2022.3173929}
    \newline
    $\bullet$ \emph{Proposed a monocular resonant beam-based integrated communication and positioning (RB-ICP) system that achieves both centimeter-level localization accuracy and high-rate data transmission.}
    \newline
    $\bullet$ \emph{Demonstrated the system attains <1 cm positioning error and 16 bit/s/Hz spectral efficiency over a 2 m range within a 15° field of view, enabling precise and high-speed AR/VR applications.}
    \\
\end{EntriesTableYear}


\end{document}
